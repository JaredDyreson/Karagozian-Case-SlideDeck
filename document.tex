\documentclass{beamer}
\usetheme{Copenhagen}

\usepackage{xcolor}
\usepackage{listings}

\begin{document}

\begin{frame}[fragile]
\frametitle{Jared Dyreson}
\begin{itemize}
		\item Graduated from Cal State Fullerton in December of 2021
		\item Former contributor to the Tuffix project at CSUF from 2018 to 2021
		\item \textcolor{red}{Something about how I can construct large scale projects and keep them in order (object oriented design patterns)}
		\item I am interested in K\&C because of how they operate, like how it's academia meets industry. 
		\item I would be a good fit for K \& C because I enjoy tinkering with new technologies and integrating these practices into my work flow.
\end{itemize}
\end{frame}

\begin{frame}
\frametitle{Project | StarbucksAutoma} 

\begin{itemize}
		\item \textcolor{red}{An automatic scheduler for the Stabucks Partner Portal that interfaces with the Google Calendar API. This sentence needs to be rewritten}
		\item This project went through a series of rewrites over the span of 3.5 years with varying degrees of success at each stage
		\item Current rendition is the fastest and there are no foreseeable major structural changes
\end{itemize}

\end{frame}

\begin{frame}
\frametitle{How it works | StarbucksAutoma}
\begin{itemize}
		\item When logging into the portal, the user needs to provide three pieces of information:
				\begin{itemize}
						\item Username
						\item Two factor authentication answer
						\item Password
				\end{itemize}
		\item These items are stored in a JSON file that has special permissions set that only root can read and stored in /etc/StarbucksAutoma
		\item Credentials are loaded into environment variables that only the current process can see
		\item User interaction is simulated using the selenium-wire module, which also gives the option to note outbound GET/POST requests
\end{itemize}
\end{frame}

\begin{frame}
\frametitle{How it works (continued) | StarbucksAutoma}
\begin{itemize}
		\item Once logged in, the program will wait until all requests are received from the internal API
		\item A filter is set that only certain requests will be stored
		\item The JSON response is then parsed and sent to the event handler
		\item The event handler will form the event packet into a format that the Google Calendar API can understand
		\item Requests are made and if successful will notify the user that the event has been added
		\item \textcolor{red}{Events can be viewed in the integrated iOS calendar}
\end{itemize}
\end{frame}

\begin{frame}

\begin{center}
Questions?
\end{center}

\end{frame}

\end{document}

